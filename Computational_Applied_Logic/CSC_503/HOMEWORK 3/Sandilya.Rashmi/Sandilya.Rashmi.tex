\documentclass{article}
% Package and macro definitions for CSC 503
% Originally prepared August 23, 2012 by Jon Doyle

%%% Page dimensions
\setlength{\oddsidemargin}{0in}
\setlength{\evensidemargin}{0in}
\setlength{\topmargin}{0in}
\setlength{\textheight}{9in}
\setlength{\textwidth}{6.5in}
\setlength{\headheight}{0in}
\setlength{\headsep}{0in}
\setlength{\footskip}{0.5in}

%%% Font and symbol definition packages
\usepackage{times} 
\usepackage{helvet} 
\usepackage{alltt}
\usepackage{amsfonts, amsmath, amsthm}
\usepackage{amssymb}
\usepackage{stmaryrd}

%%% TikZ diagramming package
\usepackage{tikz}
\usetikzlibrary{arrows,automata}

%%% The modified Sellinger fitch.sty file
\input{fitchhr.sty}

\newcommand{\Z}{\mathbb{Z}}
\newcommand{\Q}{\mathbb{Q}}
\newcommand{\R}{\mathbb{R}}
\newcommand{\N}{\mathbb{N}}
\def\land{\wedge}
\def\lor{\vee}
\def\implies{\rightarrow}
\def\iff{\leftrightarrow}
\def\turn{\vdash}
\def\lrturn{\dashv\vdash}
\def\Cn{\text{Cn}}
\def\Th{\text{Th}}
\def\defeq{\stackrel{\rm def}{=}}

%%% The environment for providing answers to problems
\newenvironment{answer}%
{\par\noindent\textbf{Answer}\par\noindent}%
{}

\def\Sometime{\mathord{\mathsf{F}}}
\def\Forever{\mathord{\mathsf{G}}}
\def\Next{\mathord{\mathsf{O}}}
\def\NextX{\mathord{\mathsf{X}}}
\def\Until{\mathrel{\mathsf{U}}}
\def\Release{\mathrel{\mathsf{R}}}
\def\WeakUntil{\mathrel{\mathsf{W}}}
\def\Before{\mathrel{\mathsf{B}}}

\def\True{\mathord{\mathsf{true}}}

\def\All{\mathord{\mathsf{A}}}
\def\Exists{\mathord{\mathsf{E}}}
\def\Every{\mathord{\mathsf{E}}}

\def\True{\mathord{\mathtt{true}}}
\def\False{\mathord{\mathtt{false}}}

\def\If{\mathrel{\mathtt{if}}}
\def\Else{\mathrel{\mathtt{else}}}
\def\While{\mathrel{\mathtt{while}}}
\def\IfElse#1#2#3{\If #1 \ \{ #2 \} \Else \{ #3 \}}
\def\Whiledo#1#2{\While #1\ \{ #2 \}}
\def\Hcond#1{\llparenthesis #1 \rrparenthesis}
\def\Hoare#1#2#3{\Hcond{#1} \mathrel{#2} \Hcond{#3}}

\def\parmodels{\mathrel{\models_{\textup{par}}}}
\def\totmodels{\mathrel{\models_{\textup{tot}}}}
\def\parturn{\mathrel{\turn_{\textup{par}}}}
\def\totturn{\mathrel{\turn_{\textup{tot}}}}


\def\unityid{rsandil}

\begin{document}
\begin{center}
  {\LARGE CSC 503 Homework Assignment 3}\\[1pc]
  Out: September 4, 2015 \\
  Due: September 11, 2015 \\
  \unityid
\end{center}

\begin{enumerate}

\item {[20 points]} Construct a formula $\phi$ in \textbf{DNF} to
  match the following truth table:
  \begin{center}
    \begin{tabular}{cc|c}
      $p$ & $q$ & $\phi$ \\ \hline
      T & T & T \\
      T & F & F \\
      F & T & T \\
      F & F & F
    \end{tabular}
  \end{center}
  
    \begin{answer}
  	From the above truth tables we can derive the DNF with following steps
  	\begin{enumerate}
  	  \item Lines with truth value as '$T$' are $Line1$ and $Line3$.
  	  \item Thus $\phi$ can be represented in DNF by conjoining two lines  $(Line1 \lor Line3)$. Except for these valuation there are no other valuation for $\phi$ as $T$.
  	  \item After representing each line in literals, formula becomes $\phi : (p \land q) \lor (\neg p
  	  \land q)$
	\end{enumerate}
	
	Thus the DNF to match the given truth table is $(p \land q) \lor (\neg p
  	  \land q)$
  \end{answer}

\item {[20 points]} Construct a formula $\phi$ in \textbf{CNF} to
  match the following truth table:
  \begin{center}
    \begin{tabular}{ccc|c}
      $p$ & $q$ & $r$ & $\phi$ \\ \hline
      T & T & T & T\\
      T & T & F & F\\
      T & F & T & F\\
      T & F & F & F\\
      F & T & T & F\\
      F & T & F & T\\
      F & F & T & T\\
      F & F & F & T
    \end{tabular}
  \end{center}
  
    \begin{answer}
  	\begin{enumerate}
  		\item Lines with truth value as '$F$' are $Line2$, $Line3$, $Line4$ and $Line5$
  		\item Thus $\phi$ can be represented in CNF as $(\neg Line2 \land
  		\neg Line3 \land \neg Line4 \land \neg Line5)$ which means that except for these line values, other values evaluates to $\phi$ as '$T$' 
  		\item Representing each row by the corresponding values of the
  		literals, the formula is $\phi : (\neg (p \land q \land \neg r)
  		\land \neg(p \land \neg q \land r) \land \neg( p \land \neg q \land
  		\neg r) \land \neg(\neg p \land q \land r))$
  		\item Using De'Morgans Law, the above formula can be reduced to $\phi: ((\neg
  		p \lor \neg q \lor r) \land (\neg p \lor q \lor \neg r) \land (\neg p \lor q
  		\lor r) \land (p \lor \neg q \lor \neg r))$
  	\end{enumerate}
  	
  	Thus the CNF for the given truth table is $((\neg
  		p \lor \neg q \lor r) \land (\neg p \lor q \lor \neg r) \land (\neg p \lor q
  		\lor r) \land (p \lor \neg q \lor \neg r))$
  \end{answer}
  
\item {[30 points]} Consider the atomic sentences
  \begin{center}
    \begin{tabular}{rcl}
      $p$ &=& The cow jumped over the moon. \\
      $q$ &=& The little dog laughed. \\
      $r$ &=& The dish ran away with the spoon. \\
    \end{tabular}
  \end{center}
  Using these, form the three complex statements
  \begin{enumerate}
  \item If the cow jumped over the moon and the little dog laughed,
    then the dish ran away with the spoon.
  \item If the little dog laughed, then the cow jumped over the moon.
  \item If the dish ran away with the spoon, then the little dog
    laughed.
  \end{enumerate}
  Show that these three complex statements are logically independent
  of each other by providing, for each of these three complex
  statements, a truth assignment to the atomic sentences that makes
  the complex statement false and the other complex statements true.
  
  \begin{answer}
  The three complex sentences can be represented as follows:
  
  1. $(p \land q) \implies r$
  
  2. $q \implies p$
  
  3. $ r \implies q$
  
 \textbf{Proof for 1st complex sentence}
  
  To prove : if  $(p \land q) \implies r$ is false then both $q \implies p$ and $ r \implies q$ are true
  
  Let us assume that  $(p \land q) \implies r$ \textbf{is false}
  
  This means $(p \land q)$ has to be '$T$'  and $r$ has to be '$F$'  
  
  Since $(p \land q)$ is '$T$', it means $p$ has to be '$T$' and $q$ has to be '$T$'
  
  Thus $q \implies p$  \textbf{is True}
  
  And $ r \implies q$  \textbf{is True}
  
   \textbf{Proof for 2nd complex sentence}
   
   To prove : if  $q \implies p$  is false then both $(p \land q) \implies r$ and $ r \implies q$ are true
   
   Let us assume that $q \implies p$ \textbf{is false}
   
   This means  $q$ has to be '$T$'  and $p$ has to be '$F$'  
   
   Thus  $(p \land q)$ has to be '$F$'
   
   Now Since $(p \land q)$ is '$F$', we can conclude that $(p \land q) \implies r$ \textbf{is True}
   
 Also since $q$ is '$T$', we can say that $ r \implies q$  \textbf{is True}
 
  \textbf{Proof for 3rd complex sentence}
  
   To prove : if  $ r \implies q$  is false then both $(p \land q) \implies r$ and $q \implies p$ are true
  
   Let us assume that $ r \implies q$ \textbf{is false}
   
   This means that $r$ has to be '$T$' and $q$ has to be '$F$'
   
   Now since $r$ is '$T$', we can say that  $(p \land q) \implies r$ \textbf{is True}
   
   Also since $q$ is '$F$', we can say that $q \implies p$  \textbf{is True}
  
  Thus we can say that the three complex sentences are logically independent of each other.
  
  
  
  \end{answer}
  

\item {[30 points]} Apply algorithm HORN from page 66 of the textbook
  to the following Horn formula.
  \begin{displaymath}
    \begin{array}{ll}
      (\top \implies q) &\land \\
      (\top \implies s) &\land \\
      (w \implies \bot) &\land \\
      (p \land q \land s \implies \bot) &\land \\
      (v \implies s) &\land \\
      (\top \implies r) &\land \\
      (r \implies p) &\land \\
      (p \land s \implies s)
    \end{array}
  \end{displaymath}
  Your answer should list propositional letters in the order in which
  they are marked as well as giving the overall answer.
  
  \begin{answer}
	\begin{enumerate}
	  \item Mark all occurances of $\top$.
	  	
	  \item Mark $q, s, r$ from $(\top \implies q), (\top \implies s),
	  (\top \implies r)$
	  
		\begin{displaymath}
    		\begin{array}{ll}
   				      (\top \implies \textbf{q}) &\land \\
                      (\top \implies \textbf{s}) &\land \\
                      (w \implies \bot) &\land \\
                      (p \land \textbf{q} \land \textbf{s} \implies \bot) &\land \\
                      (v \implies \textbf{s}) &\land \\
                      (\top \implies \textbf{r}) &\land \\
                      (\textbf{r} \implies p) &\land \\
                      (p \land \textbf{s} \implies \textbf{s})
    		\end{array}
  		\end{displaymath}
  		\item Mark $p$ from $(r \implies p)$
  		
		\begin{displaymath}
    		\begin{array}{ll}
   				      (\top \implies \textbf{q}) &\land \\
                      (\top \implies \textbf{s}) &\land \\
                      (w \implies \bot) &\land \\
                      (\textbf{p} \land \textbf{q} \land \textbf{s} \implies \bot) &\land \\
                      (v \implies \textbf{s}) &\land \\
                      (\top \implies \textbf{r}) &\land \\
                      (\textbf{r} \implies \textbf{p}) &\land \\
                      (\textbf{p} \land \textbf{s} \implies \textbf{s})
    		\end{array}
  		\end{displaymath}
  		
  		\item Mark $\bot$ from $(\textbf{p} \land \textbf{q} \land \textbf{s}
  		\implies \bot)$
  		
	\end{enumerate}
	
	The order of propositional letters is $q, s, r, p,\bot$. 
	Since we have marked $\bot$, the horn formula is \textbf{'Unsatisfiable'}.
\end{answer}

\end{enumerate}
\end{document}
