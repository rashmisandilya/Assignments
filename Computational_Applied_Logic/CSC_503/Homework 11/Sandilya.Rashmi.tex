\documentclass{article}
% Package and macro definitions for CSC 503
% Originally prepared August 23, 2012 by Jon Doyle

%%% Page dimensions
\setlength{\oddsidemargin}{0in}
\setlength{\evensidemargin}{0in}
\setlength{\topmargin}{0in}
\setlength{\textheight}{9in}
\setlength{\textwidth}{6.5in}
\setlength{\headheight}{0in}
\setlength{\headsep}{0in}
\setlength{\footskip}{0.5in}

%%% Font and symbol definition packages
\usepackage{times} 
\usepackage{helvet} 
\usepackage{alltt}
\usepackage{amsfonts, amsmath, amsthm}
\usepackage{amssymb}
\usepackage{stmaryrd}

%%% TikZ diagramming package
\usepackage{tikz}
\usetikzlibrary{arrows,automata}

%%% The modified Sellinger fitch.sty file
\input{fitchhr.sty}

\newcommand{\Z}{\mathbb{Z}}
\newcommand{\Q}{\mathbb{Q}}
\newcommand{\R}{\mathbb{R}}
\newcommand{\N}{\mathbb{N}}
\def\land{\wedge}
\def\lor{\vee}
\def\implies{\rightarrow}
\def\iff{\leftrightarrow}
\def\turn{\vdash}
\def\lrturn{\dashv\vdash}
\def\Cn{\text{Cn}}
\def\Th{\text{Th}}
\def\defeq{\stackrel{\rm def}{=}}

%%% The environment for providing answers to problems
\newenvironment{answer}%
{\par\noindent\textbf{Answer}\par\noindent}%
{}

\def\Sometime{\mathord{\mathsf{F}}}
\def\Forever{\mathord{\mathsf{G}}}
\def\Next{\mathord{\mathsf{O}}}
\def\NextX{\mathord{\mathsf{X}}}
\def\Until{\mathrel{\mathsf{U}}}
\def\Release{\mathrel{\mathsf{R}}}
\def\WeakUntil{\mathrel{\mathsf{W}}}
\def\Before{\mathrel{\mathsf{B}}}

\def\True{\mathord{\mathsf{true}}}

\def\All{\mathord{\mathsf{A}}}
\def\Exists{\mathord{\mathsf{E}}}
\def\Every{\mathord{\mathsf{E}}}

\def\True{\mathord{\mathtt{true}}}
\def\False{\mathord{\mathtt{false}}}

\def\If{\mathrel{\mathtt{if}}}
\def\Else{\mathrel{\mathtt{else}}}
\def\While{\mathrel{\mathtt{while}}}
\def\IfElse#1#2#3{\If #1 \ \{ #2 \} \Else \{ #3 \}}
\def\Whiledo#1#2{\While #1\ \{ #2 \}}
\def\Hcond#1{\llparenthesis #1 \rrparenthesis}
\def\Hoare#1#2#3{\Hcond{#1} \mathrel{#2} \Hcond{#3}}

\def\parmodels{\mathrel{\models_{\textup{par}}}}
\def\totmodels{\mathrel{\models_{\textup{tot}}}}
\def\parturn{\mathrel{\turn_{\textup{par}}}}
\def\totturn{\mathrel{\turn_{\textup{tot}}}}


\def\unityid{rsandil}

\usepackage{amsfonts, amsmath, amsthm}
\usepackage{tikz}
\usetikzlibrary{arrows,automata}

\usepackage{stmaryrd}

\def\Sometime{\mathord{\mathsf{F}}}
\def\Forever{\mathord{\mathsf{G}}}
\def\Next{\mathord{\mathsf{O}}}
\def\NextX{\mathord{\mathsf{X}}}
\def\Until{\mathrel{\mathsf{U}}}
\def\Release{\mathrel{\mathsf{R}}}
\def\WeakUntil{\mathrel{\mathsf{W}}}
\def\Before{\mathrel{\mathsf{B}}}
\def\True{\mathord{\mathsf{true}}}
\def\All{\mathord{\mathsf{A}}}
\def\Exists{\mathord{\mathsf{E}}}
\def\Every{\mathord{\mathsf{E}}}

\def\True{\mathord{\mathtt{true}}}
\def\False{\mathord{\mathtt{false}}}
\def\If{\mathrel{\mathtt{if}}}
\def\Else{\mathrel{\mathtt{else}}}
\def\While{\mathrel{\mathtt{while}}}
\def\IfElse#1#2#3{\If #1 \ \{ #2 \} \Else \{ #3 \}}
\def\Whiledo#1#2{\While #1\ \{ #2 \}}
\def\Hcond#1{\llparenthesis #1 \rrparenthesis}
\def\Hoare#1#2#3{\Hcond{#1} \mathrel{#2} \Hcond{#3}}

\def\parmodels{\mathrel{\models_{\textup{par}}}}
\def\totmodels{\mathrel{\models_{\textup{tot}}}}
\def\parturn{\mathrel{\turn_{\textup{par}}}}
\def\totturn{\mathrel{\turn_{\textup{tot}}}}



\begin{document}
\begin{center}
  {\LARGE CSC 503 Homework Assignment 11}\\[1pc]
  Out: October 23, 2015 \\
  Due: October 30, 2015 \\
  \unityid
\end{center}


In the following proofs about programs, use the natural deduction
proof environment, the \verb+\Hcond+ macro, and the Hoare-logic
justifications \verb+\Implied{}+, \verb+\Assignment{}+,
\verb+\IfStatement{}+, \verb+\PartialWhile{}+, \\
\verb+\InvariantGuard{}+, and \verb+\TotalWhile{}+ to present proofs
of partial or total correctness.  If you are not using LaTeX, you can
substitute double parentheses for the Hoare condition delimiters
$\Hcond{~}$.  For example, we can rewrite the proof in Example 4.13.1
on page 277 of the textbook as
  \begin{displaymath}
    \begin{nd}
      \have{} {\Hcond{y = 5}}
      \have{} {\Hcond{y + 1 = 6}}         \Implied{}
      \have{} {\texttt{\textbf{x = y + 1;}}}   
      \have{} {\Hcond{x = 6}}          \Assignment{}
    \end{nd}
  \end{displaymath}
  The standard stmaryrd package is also used in these macros.

\begin{enumerate}

\item \textbf{[20 points]} Recall that the arithmetic expressions in
  the simple textbook programming language only refer to integers.
  Use the proof rule for assignment and integer arithmetic entailment
  as appropriate to show the validity of
  \begin{displaymath}
    \parturn \Hoare{x > 1 \land y > 0}{P}{x > 1 \land y \ge 2}.
  \end{displaymath}
where $P$ is the program 
  \begin{displaymath}
    \begin{nd}
      \have{} {\texttt{\textbf{a = 2 * x;}}}   
      \have{} {\texttt{\textbf{y = x + a;}}}          
      \have{} {\texttt{\textbf{y = y - x;}}}          
    \end{nd}
  \end{displaymath}
  
\begin{answer}
  	\begin{displaymath}
    \begin{nd}
      \have{} {\Hcond{x > 1 \land y > 0}} 
      \have{} {\Hcond{x > 1 \land 2x \ge 2}} \Implied{} 
      \have{} {\texttt{\textbf{a = 2 * x}}} 
      \have{} {\Hcond{x > 1 \land a \ge 2}} \Assignment{} 
      \have{} {\texttt{\textbf{y = x + a;}}}   
      \have{} {\Hcond{x > 1 \land y - x \ge 2}} \Assignment{} 
      \have{} {\texttt{\textbf{y = y - x;}}}
      \have{} {\Hcond{x > 1 \land y \ge 2}} \Assignment{}          
    \end{nd}
  \end{displaymath}
  \end{answer}
  
\item \textbf{[40 points]} Prove the validity of the sequent
  $\parturn \Hoare{y > 0}{Q}{w = \max(y,z)}$ where $\max(y,z)$ is the
  largest number of $y$ and $z$, and where the code of $Q$ is given by
  \begin{displaymath}
    \begin{nd}
      \have{} {\texttt{\textbf{x = 0;}}}   
      \have{} {\texttt{\textbf{if (x < y) \{}}}   
      \open
      \have{} {\texttt{\textbf{if (y < z) \{}}}   
      \open
      \have{} {\texttt{\textbf{w = z;}}}          
      \close
      \have{} {\texttt{\textbf{\} else \{}}}                   
      \open
      \have{} {\texttt{\textbf{w = y;}}}   
      \close
      \have{} {\texttt{\textbf{\};}}}                  
      \close
      \have{} {\texttt{\textbf{\} else \{}}}                   
      \open
      \have{} {\texttt{\textbf{w = x;}}}   
      \close
      \have{} {\texttt{\textbf{\}}}}                  
    \end{nd}
  \end{displaymath}  
  \begin{answer}
  \begin{displaymath}
    \begin{nd}
      \have{} {\Hcond{ y > 0}}
      \have{} {\Hcond{(0 < y \implies ((y < z) \implies z = max(y,z))\land(\neg(y < z) \implies y = max(y,z)))}  \\ 
      {\land(\neg(0 < y) \implies 0 = max(y,z))}}\Implied{}
      \have{} {\texttt{\textbf{x = 0;}}}
      \have{} {\Hcond{(x < y \implies ((y < z) \implies z = max(y,z))\land(\neg(y < z) \implies y = max(y,z)))}  \\ 
      {\land(\neg(x < y) \implies x = max(y,z))}}\Assignment{}
      \have{} {\texttt{\textbf{if (x < y) \{}}}
      \have{} {\Hcond{(y < z \implies z = max(y,z) \land \neg(y < z) \implies y =
      max(y,z)}} \IfStatement{}   
      \open
      \have{} {\texttt{\textbf{if (y < z) \{}}}   
      \open
      \have{} {\Hcond{z = \max(y,z)}} \IfStatement{}  
      \have{} {\texttt{\textbf{w = z;}}}
      \have{} {\Hcond{w = \max(y,z)}} \Assignment{}            
      \close
      \have{} {\texttt{\textbf{\} else \{}}}                   
      \open
      \have{} {\Hcond{y = \max(y,z)}} \IfStatement{}  
      \have{} {\texttt{\textbf{w = y;}}} 
      \have{} {\Hcond{w = \max(y,z)}} \Assignment{}    
      \close
      \have{} {\texttt{\textbf{\};}}}                  
      \close
      \have{} {\Hcond{w = \max(y,z)}} \IfStatement{} 
      \have{} {\texttt{\textbf{\} else \{}}}                   
      \open
      \have{} {\Hcond{x = \max(y,z)}} \IfStatement{}  
      \have{} {\texttt{\textbf{w = x;}}}   
      \have{} {\Hcond{w = \max(y,z)}} \Assignment{}  
      \close
      \have{} {\texttt{\textbf{\}}}}    
      \have{} {\Hcond{w = \max(y,z)}} \IfStatement{}               
    \end{nd}
  \end{displaymath}
  \end{answer}
NOTE: In line number 2 and 4 vertical line paranthesis is meant to enclose the complete statement split in multiline(2). Formatting in latex caused the problem.
\item \textbf{[40 points]} Prove the validity of the sequent
  $\parturn \Hoare{w = x \land x \ge 0}{R}{z = x \cdot y}$ where the
  code of $R$ is given by
  \begin{displaymath}
    \begin{nd}
      \have{} {\texttt{\textbf{z = 0;}}}   
      \have{} {\texttt{\textbf{while (w != 0) \{}}}   
      \open
      \have{} {\texttt{\textbf{z = z + y};}}   
      \have{} {\texttt{\textbf{w = w - 1};}}   
      \close
      \have{} {\texttt{\textbf{\}}}}                  
    \end{nd}
  \end{displaymath}
\end{enumerate}
\begin{answer}
	\begin{displaymath}
    \begin{nd}
      \have{} {\Hcond{w = x \land x \ge 0}} 
      \have{} {\Hcond{x = w}} \Implied{}
      \have{} {\Hcond{0 = (x - w)y}}\Implied{}
      \have{} {\texttt{\textbf{z = 0;}}}   
      \have{} {\Hcond{z = (x - w)y}}\Assignment{}
      \have{} {\texttt{\textbf{while (w != 0) \{}}}   
      \open
      \have{} {\Hcond{z = (x - w)y \land \neg (w != 0)}} \InvariantGuard{}
      \have{} {\Hcond{z + y = (x - (w - 1)) y}} \Implied{}
      \have{} {\texttt{\textbf{z = z + y};}} 
      \have{} {\Hcond{z = (x - (w - 1)) y}} \Assignment{} 
      \have{} {\texttt{\textbf{w = w - 1};}}  
      \have{} {\Hcond{z = (x - w) y}} \Assignment{}
      \close
      \have{} {\texttt{\textbf{\}}}} 
      \have{} {\Hcond{z = (x - w)\cdot y \land \neg (w != 0)}} \PartialWhile{}
      \have{} {\Hcond{z = x \cdot y}} \Implied{}
    \end{nd}
  \end{displaymath}
\end{answer}
\end{document}
