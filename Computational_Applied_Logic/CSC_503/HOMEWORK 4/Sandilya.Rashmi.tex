\documentclass{article}
% Package and macro definitions for CSC 503
% Originally prepared August 23, 2012 by Jon Doyle

%%% Page dimensions
\setlength{\oddsidemargin}{0in}
\setlength{\evensidemargin}{0in}
\setlength{\topmargin}{0in}
\setlength{\textheight}{9in}
\setlength{\textwidth}{6.5in}
\setlength{\headheight}{0in}
\setlength{\headsep}{0in}
\setlength{\footskip}{0.5in}

%%% Font and symbol definition packages
\usepackage{times} 
\usepackage{helvet} 
\usepackage{alltt}
\usepackage{amsfonts, amsmath, amsthm}
\usepackage{amssymb}
\usepackage{stmaryrd}

%%% TikZ diagramming package
\usepackage{tikz}
\usetikzlibrary{arrows,automata}

%%% The modified Sellinger fitch.sty file
\input{fitchhr.sty}

\newcommand{\Z}{\mathbb{Z}}
\newcommand{\Q}{\mathbb{Q}}
\newcommand{\R}{\mathbb{R}}
\newcommand{\N}{\mathbb{N}}
\def\land{\wedge}
\def\lor{\vee}
\def\implies{\rightarrow}
\def\iff{\leftrightarrow}
\def\turn{\vdash}
\def\lrturn{\dashv\vdash}
\def\Cn{\text{Cn}}
\def\Th{\text{Th}}
\def\defeq{\stackrel{\rm def}{=}}

%%% The environment for providing answers to problems
\newenvironment{answer}%
{\par\noindent\textbf{Answer}\par\noindent}%
{}

\def\Sometime{\mathord{\mathsf{F}}}
\def\Forever{\mathord{\mathsf{G}}}
\def\Next{\mathord{\mathsf{O}}}
\def\NextX{\mathord{\mathsf{X}}}
\def\Until{\mathrel{\mathsf{U}}}
\def\Release{\mathrel{\mathsf{R}}}
\def\WeakUntil{\mathrel{\mathsf{W}}}
\def\Before{\mathrel{\mathsf{B}}}

\def\True{\mathord{\mathsf{true}}}

\def\All{\mathord{\mathsf{A}}}
\def\Exists{\mathord{\mathsf{E}}}
\def\Every{\mathord{\mathsf{E}}}

\def\True{\mathord{\mathtt{true}}}
\def\False{\mathord{\mathtt{false}}}

\def\If{\mathrel{\mathtt{if}}}
\def\Else{\mathrel{\mathtt{else}}}
\def\While{\mathrel{\mathtt{while}}}
\def\IfElse#1#2#3{\If #1 \ \{ #2 \} \Else \{ #3 \}}
\def\Whiledo#1#2{\While #1\ \{ #2 \}}
\def\Hcond#1{\llparenthesis #1 \rrparenthesis}
\def\Hoare#1#2#3{\Hcond{#1} \mathrel{#2} \Hcond{#3}}

\def\parmodels{\mathrel{\models_{\textup{par}}}}
\def\totmodels{\mathrel{\models_{\textup{tot}}}}
\def\parturn{\mathrel{\turn_{\textup{par}}}}
\def\totturn{\mathrel{\turn_{\textup{tot}}}}


\def\unityid{rsandil}

\begin{document}
\begin{center}
  {\LARGE CSC 503 Homework Assignment 4}\\[1pc]
  Out: September 9, 2015 \\
  Due: September 16, 2015 \\
  \unityid
\end{center}

Unless directed otherwise, follow the convention of the text and
assume that $a,b,c,d,e$ are constant symbols, $f,g,h$ are function
symbols, and $w,u,v,x,y,z$ are variable symbols.

\begin{enumerate}

\item Use the predicates
  \begin{center}
    \begin{tabular}{rl}
      $C(x,y)$ : & $x$ is a champion of $y$ \\
      $F(x,y)$ : & $x$ is a fan of $y$ \\
      $Q(x,y)$ : & $x$ is the quarterback of $y$ \\
      $R(x,y)$ : & $x$ is a rival of $y$ \\
      $S(x,y)$ : & $x$ is the sister of $y$ \\
      $T(x)$   : & $x$ is a team \\
    \end{tabular}
  \end{center}
and the constant (nullary function) symbols
  \begin{center}
    \begin{tabular}{rl}
      $s$ : & Serena \\
      $t$ : & Tom \\
    \end{tabular}
  \end{center}
to translate the following English sentences into predicate logic.
You are not allowed to use any predicate, function, or constant
symbols other than the above.
\begin{enumerate}
\item {[5 points]} Serena is a champion.
\begin{answer}
     \begin{displaymath}
           \exists x(C(s, x))
       \end{displaymath}
\end{answer}
\item {[5 points]} Any team that has Serena for a quarterback has Tom
  for a fan.
  \begin{answer}
     \begin{displaymath}
           \forall x ((T(x)\land Q(s,x)) \implies F(t,x))
       \end{displaymath}
\end{answer}
\item {[5 points]} Tom is a fan of every champion.
  \begin{answer}
     \begin{displaymath}
           \forall x \exists y(C(x,y)\implies F(t,x))
       \end{displaymath}
\end{answer}
\item {[5 points]} Tom is a fan of Tom.
  \begin{answer}
     \begin{displaymath}
       F(t,t)
       \end{displaymath}
\end{answer}
\item {[5 points]} Every team has a fan.
  \begin{answer}
     \begin{displaymath}
       \forall x(T(x) \implies \exists y(F(y,x)))
       \end{displaymath}
\end{answer}
\item {[5 points]} All champions are rivals.
  \begin{answer}
     \begin{displaymath}
       \forall x \forall y (C(x,y) \implies \exists w (R(x,w)))
       \end{displaymath}
\end{answer}
\item {[5 points]} Only teams have rivals.
  \begin{answer}
     \begin{displaymath}
   \forall x \forall y(R(x,y)\implies(T(y))
        \end{displaymath}
\end{answer}  
\item {[5 points]} All rivals are teams that have Tom for a quarterback.
  \begin{answer}
     \begin{displaymath}
   \forall x \forall y((R(x,y)\implies(T(x)\land Q(t,x)))
        \end{displaymath}
\end{answer} 
\item {[5 points]} Some sister of some champion is a champion.
  \begin{answer}
     \begin{displaymath}
   \exists x \exists y((S(x,y)\land \exists z(C(y,z)) \land \exists w(C(x,w)))
        \end{displaymath}
\end{answer} 
\item {[5 points]} Every sister of every champion is a champion.
\begin{answer}
 \begin{displaymath}
 \forall x \forall y((S(x,y)\land \exists z(C(y,z)) \land \exists w(C(x,w)))
         \end{displaymath}
\end{answer}
\end{enumerate}

\item Let $c$ and $d$ be constants, $f$ a function symbol with two
  arguments, $g$ a function symbol with three arguments, $h$ a
  function symbol with one argument, $P$ a predicate symbol with two
  arguments, and $Q$ a predicate symbol with three arguments.
  Indicate, for each of the following strings, which strings are
  formulas in predicate logic, and specify a reason for failure for
  strings which are not.
  \begin{enumerate}
  \item {[5 points]} $\forall x\ Q(f(d,y),g(h(c,x),d,y),x)$
  \begin{answer}
  This formula is \textbf{invalid} as function $h$ has arity of one but here it has been incorrectly used to accept two arguments.
  \end{answer}
  \item {[5 points]} $\forall x\ P(x,c) \lor g(f(d,x),h(y),y)$
   \begin{answer}
  This formula is a \textbf{invalid} since left hand side of logical connective $\lor$ is a formula whereas right hand side is a term.
    \end{answer}
  \item {[5 points]} $\forall x (Q(z,z,z) \implies P(h(P(z,z)),z))$
  \begin{answer}
  This formula is \textbf{invalid} as $P(z,z)$ is not a term and so all the parameters to the function $h$ are not terms.
  \end{answer}
  \item {[5 points]} $Q(h(h(h(c))),d,\neg f(d,d)) \implies P(c,c)$
    \begin{answer}
  This formula is a \textbf{invalid} formula in predicate logic as negation of $f(d,d)$ is not a term
   \end{answer}
  \item {[5 points]} $\forall x \forall y \exists z\ P(c,d,c)$
      \begin{answer}
  This formula is \textbf{invalid} formula in predicate logic since the
  	predicate $P$ is incorrectly used to accept three arguments. Predicate $P$ accepts only two arguments.
   \end{answer}
  \end{enumerate}

\item Let $P$ be a predicate symbol with arity 2, and let $\phi$ be
  the formula
  \begin{displaymath}
    \forall y\ [(\neg P(y,x) \lor P(y,z)) \land \exists y \forall z\ P(y,z)]
  \end{displaymath}
  \begin{enumerate}
  \item {[5 points]} Indicate, for each occurrence of each variable in
    $\phi$, whether that occurrence is free or bound.
        \begin{answer}
    	The variables in $\phi$ include $x, y$ and $z$.
    	The highlighted variables in $\phi$ are free.
    	
    	\begin{displaymath}
    		\forall y\ [(\neg P(y,\textbf{x}) \lor P(y,\textbf{z})) \land \exists y \forall z\ P(y,z)]
  		\end{displaymath}
  		 The rest of the variables are bounded.
    \end{answer}
  \item {[5 points]} List all variables which occur both free and
    bound in $\phi$.
    \begin{answer}
    	The variable $z$ in $\phi$ are both free and bound. $z$ is bounded by $\exists y \forall z\ P(y,z)$ on right hand side.
    	$z$ is free left hand side.
    \end {answer}
  \item {[5 points]} Compute $\phi[t/x]$ for $t = g(f(g(y,y)),z)$.  Is
    $t$ free for $x$ in $\phi$?
    \begin{answer}
    	$t$ is not free for $x$ in $\phi$ as the free instance of $x$ on replacement with $t$ will be become bounded as $t$ is function of $y$.Thus $\phi[t/x]$ will remain $\phi$. 
    \end{answer}
  \item {[5 points]} Compute $\phi[t/y]$ for $t = g(f(g(y,y)),z)$ Is
    $t$ free for $y$ in $\phi$?
    \begin{answer}
     $t$ is not free for $y$ in $\phi$ as there is no free occurrence of $y$ in $\phi$ to be
    	replaced by $t$. Thus $\phi[t/x]$ will remain $\phi$.
    \end{answer}
  \item {[5 points]} Compute $\phi[t/z]$ for $t = g(f(g(y,y)),z)$ Is
    $t$ free for $z$ in $\phi$?
   \begin{answer}
    	$t$ is not free for $z$ in $\phi$ as the free instance of $z$ on replacement with $t$ will become bounded since $t$ is a function of $y$ .Thus $\phi[t/z]$ will remain $\phi$. 
    \end{answer}
  \end{enumerate}

\end{enumerate}
\end{document}
