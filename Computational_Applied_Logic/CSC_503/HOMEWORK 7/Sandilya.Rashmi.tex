\documentclass{article}
% Package and macro definitions for CSC 503
% Originally prepared August 23, 2012 by Jon Doyle

%%% Page dimensions
\setlength{\oddsidemargin}{0in}
\setlength{\evensidemargin}{0in}
\setlength{\topmargin}{0in}
\setlength{\textheight}{9in}
\setlength{\textwidth}{6.5in}
\setlength{\headheight}{0in}
\setlength{\headsep}{0in}
\setlength{\footskip}{0.5in}

%%% Font and symbol definition packages
\usepackage{times} 
\usepackage{helvet} 
\usepackage{alltt}
\usepackage{amsfonts, amsmath, amsthm}
\usepackage{amssymb}
\usepackage{stmaryrd}

%%% TikZ diagramming package
\usepackage{tikz}
\usetikzlibrary{arrows,automata}

%%% The modified Sellinger fitch.sty file
% These are Selinger's fitch.sty macros modified by Jon Doyle to
% conform with the nomenclature of inference rules in the Huth and
% Ryan textbook.  All the license and disclaimers of fitch.sty are
% maintained.  Extensions copyright (C) Jon Doyle, August 23, 2012.

% Macros for Fitch-style natural deduction. 
% Author: Peter Selinger, University of Ottawa
% Created: Jan 14, 2002
% Modified: Feb 8, 2005
% Version: 0.5
% Copyright: (C) 2002-2005 Peter Selinger
% Filename: fitch.sty
% Documentation: fitchdoc.tex
% URL: http://quasar.mathstat.uottawa.ca/~selinger/fitch/

% License:
%
% This program is free software; you can redistribute it and/or modify
% it under the terms of the GNU General Public License as published by
% the Free Software Foundation; either version 2, or (at your option)
% any later version.
%
% This program is distributed in the hope that it will be useful, but
% WITHOUT ANY WARRANTY; without even the implied warranty of
% MERCHANTABILITY or FITNESS FOR A PARTICULAR PURPOSE. See the GNU
% General Public License for more details.
%
% You should have received a copy of the GNU General Public License
% along with this program; if not, write to the Free Software Foundation, 
% Inc., 59 Temple Place, Suite 330, Boston, MA 02111-1307, USA.

% USAGE EXAMPLE:
% 
% The following is a simple example illustrating the usage of this
% package.  For detailed instructions and additional functionality, see
% the user guide, which can be found in the file fitchdoc.tex.
% 
% \[
% \begin{nd}
%   \hypo{1}  {P\vee Q}   
%   \hypo{2}  {\neg Q}                         
%   \open                              
%   \hypo{3a} {P}
%   \have{3b} {P}        \r{3a}
%   \close                   
%   \open
%   \hypo{4a} {Q}
%   \have{4b} {\neg Q}   \r{2}
%   \have{4c} {\bot}     \ne{4a,4b}
%   \have{4d} {P}        \be{4c}
%   \close                             
%   \have{5}  {P}        \oe{1,3a-3b,4a-4d}                 
% \end{nd}
% \]

{\chardef\x=\catcode`\*
\catcode`\*=11
\global\let\nd*astcode\x}
\catcode`\*=11

% References

\newcount\nd*ctr
\def\nd*render{\expandafter\ifx\expandafter\nd*x\nd*base\nd*x\the\nd*ctr\else\nd*base\ifnum\nd*ctr<0\the\nd*ctr\else\ifnum\nd*ctr>0+\the\nd*ctr\fi\fi\fi}
\expandafter\def\csname nd*-\endcsname{}

\def\nd*num#1{\nd*numo{\nd*render}{#1}\global\advance\nd*ctr1}
\def\nd*numopt#1#2{\nd*numo{$#1$}{#2}}
\def\nd*numo#1#2{\edef\x{#1}\mbox{$\x$}\expandafter\global\expandafter\let\csname nd*-#2\endcsname\x}
\def\nd*ref#1{\expandafter\let\expandafter\x\csname nd*-#1\endcsname\ifx\x\relax%
  \errmessage{Undefined natdeduction reference: #1}\else\mbox{$\x$}\fi}
\def\nd*noop{}
\def\nd*set#1#2{\ifx\relax#1\nd*noop\else\global\def\nd*base{#1}\fi\ifx\relax#2\relax\else\global\nd*ctr=#2\fi}
\def\nd*reset{\nd*set{}{1}}
\def\nd*refa#1{\nd*ref{#1}}
\def\nd*aux#1#2{\ifx#2-\nd*refa{#1}--\def\nd*c{\nd*aux{}}%
  \else\ifx#2,\nd*refa{#1}, \def\nd*c{\nd*aux{}}%
  \else\ifx#2;\nd*refa{#1}; \def\nd*c{\nd*aux{}}%
  \else\ifx#2.\nd*refa{#1}. \def\nd*c{\nd*aux{}}%
  \else\ifx#2)\nd*refa{#1})\def\nd*c{\nd*aux{}}%
  \else\ifx#2(\nd*refa{#1}(\def\nd*c{\nd*aux{}}%
  \else\ifx#2\nd*end\nd*refa{#1}\def\nd*c{}%
  \else\def\nd*c{\nd*aux{#1#2}}%
  \fi\fi\fi\fi\fi\fi\fi\nd*c}
\def\ndref#1{\nd*aux{}#1\nd*end}

% Layer A

% define various dimensions (explained in fitchdoc.tex):
\newlength{\nd*dim} 
\newdimen\nd*depthdim
\newdimen\nd*hsep
\newdimen\ndindent
\ndindent=1em
% user command to redefine dimensions
\def\nddim#1#2#3#4#5#6#7#8{\nd*depthdim=#3\relax\nd*hsep=#6\relax%
\def\nd*height{#1}\def\nd*thickness{#8}\def\nd*initheight{#2}%
\def\nd*indent{#5}\def\nd*labelsep{#4}\def\nd*justsep{#7}}
% set initial dimensions
\nddim{4.5ex}{3.5ex}{1.5ex}{1em}{1.6em}{.5em}{2.5em}{.2mm}

\def\nd*v{\rule[-\nd*depthdim]{\nd*thickness}{\nd*height}}
\def\nd*t{\rule[-\nd*depthdim]{0mm}{\nd*height}\rule[-\nd*depthdim]{\nd*thickness}{\nd*initheight}}
\def\nd*i{\hspace{\nd*indent}} 
\def\nd*s{\hspace{\nd*hsep}}
\def\nd*g#1{\nd*f{\makebox[\nd*indent][c]{$#1$}}}
\def\nd*f#1{\raisebox{0pt}[0pt][0pt]{$#1$}}
\def\nd*u#1{\makebox[0pt][l]{\settowidth{\nd*dim}{\nd*f{#1}}%
    \addtolength{\nd*dim}{2\nd*hsep}\hspace{-\nd*hsep}\rule[-\nd*depthdim]{\nd*dim}{\nd*thickness}}\nd*f{#1}}

% Lists

\def\nd*push#1#2{\expandafter\gdef\expandafter#1\expandafter%
  {\expandafter\nd*cons\expandafter{#1}{#2}}}
\def\nd*pop#1{{\def\nd*nil{\gdef#1{\nd*nil}}\def\nd*cons##1##2%
    {\gdef#1{##1}}#1}}
\def\nd*iter#1#2{{\def\nd*nil{}\def\nd*cons##1##2{##1#2{##2}}#1}}
\def\nd*modify#1#2#3{{\def\nd*nil{\gdef#1{\nd*nil}}\def\nd*cons##1##2%
    {\advance#2-1 ##1\advance#2 1 \ifnum#2=1\nd*push#1{#3}\else%
      \nd*push#1{##2}\fi}#1}}

\def\nd*cont#1{{\def\nd*t{\nd*v}\def\nd*v{\nd*v}\def\nd*g##1{\nd*i}%
    \def\nd*i{\nd*i}\def\nd*nil{\gdef#1{\nd*nil}}\def\nd*cons##1##2%
    {##1\expandafter\nd*push\expandafter#1\expandafter{##2}}#1}}

% Layer B

\newcount\nd*n
\def\nd*beginb{\begingroup\nd*reset\gdef\nd*stack{\nd*nil}\nd*push\nd*stack{\nd*t}%
  \begin{array}{l@{\hspace{\nd*labelsep}}l@{\hspace{\nd*justsep}}l}}
\def\nd*resumeb{\begingroup\begin{array}{l@{\hspace{\nd*labelsep}}l@{\hspace{\nd*justsep}}l}}
\def\nd*endb{\end{array}\endgroup}
\def\nd*hypob#1#2{\nd*f{\nd*num{#1}}&\nd*iter\nd*stack\relax\nd*cont\nd*stack\nd*s\nd*u{#2}&}
\def\nd*haveb#1#2{\nd*f{\nd*num{#1}}&\nd*iter\nd*stack\relax\nd*cont\nd*stack\nd*s\nd*f{#2}&}
\def\nd*havecontb#1#2{&\nd*iter\nd*stack\relax\nd*cont\nd*stack\nd*s\nd*f{\hspace{\ndindent}#2}&}
\def\nd*hypocontb#1#2{&\nd*iter\nd*stack\relax\nd*cont\nd*stack\nd*s\nd*u{\hspace{\ndindent}#2}&}

\def\nd*openb{\nd*push\nd*stack{\nd*i}\nd*push\nd*stack{\nd*t}}
\def\nd*closeb{\nd*pop\nd*stack\nd*pop\nd*stack}
\def\nd*guardb#1#2{\nd*n=#1\multiply\nd*n by 2 \nd*modify\nd*stack\nd*n{\nd*g{#2}}}

% Layer C

\def\nd*clr{\gdef\nd*cmd{}\gdef\nd*typ{\relax}}
\def\nd*sto#1#2#3{\gdef\nd*typ{#1}\gdef\nd*byt{}%
  \gdef\nd*cmd{\nd*typ{#2}{#3}\nd*byt\\}}
\def\nd*chtyp{\expandafter\ifx\nd*typ\nd*hypocontb\def\nd*typ{\nd*havecontb}\else\def\nd*typ{\nd*haveb}\fi}
\def\nd*hypoc#1#2{\nd*chtyp\nd*cmd\nd*sto{\nd*hypob}{#1}{#2}}
\def\nd*havec#1#2{\nd*cmd\nd*sto{\nd*haveb}{#1}{#2}}
\def\nd*hypocontc#1{\nd*chtyp\nd*cmd\nd*sto{\nd*hypocontb}{}{#1}}
\def\nd*havecontc#1{\nd*cmd\nd*sto{\nd*havecontb}{}{#1}}
\def\nd*by#1#2{\ifx\nd*x#2\nd*x\gdef\nd*byt{\mbox{#1}}\else\gdef\nd*byt{\mbox{#1, \ndref{#2}}}\fi}

% multi-line macros
\def\nd*mhypoc#1#2{\nd*mhypocA{#1}#2\\\nd*stop\\}
\def\nd*mhypocA#1#2\\{\nd*hypoc{#1}{#2}\nd*mhypocB}
\def\nd*mhypocB#1\\{\ifx\nd*stop#1\else\nd*hypocontc{#1}\expandafter\nd*mhypocB\fi}
\def\nd*mhavec#1#2{\nd*mhavecA{#1}#2\\\nd*stop\\}
\def\nd*mhavecA#1#2\\{\nd*havec{#1}{#2}\nd*mhavecB}
\def\nd*mhavecB#1\\{\ifx\nd*stop#1\else\nd*havecontc{#1}\expandafter\nd*mhavecB\fi}
\def\nd*mhypocontc#1{\nd*mhypocB#1\\\nd*stop\\}
\def\nd*mhavecontc#1{\nd*mhavecB#1\\\nd*stop\\}

\def\nd*beginc{\nd*beginb\nd*clr}
\def\nd*resumec{\nd*resumeb\nd*clr}
\def\nd*endc{\nd*cmd\nd*endb}
\def\nd*openc{\nd*cmd\nd*clr\nd*openb}
\def\nd*closec{\nd*cmd\nd*clr\nd*closeb}
\let\nd*guardc\nd*guardb

% Layer D

% macros with optional arguments spelled-out
\def\nd*hypod[#1][#2]#3[#4]#5{\ifx\relax#4\relax\else\nd*guardb{1}{#4}\fi\nd*mhypoc{#3}{#5}\nd*set{#1}{#2}}
\def\nd*haved[#1][#2]#3[#4]#5{\ifx\relax#4\relax\else\nd*guardb{1}{#4}\fi\nd*mhavec{#3}{#5}\nd*set{#1}{#2}}
\def\nd*havecont#1{\nd*mhavecontc{#1}}
\def\nd*hypocont#1{\nd*mhypocontc{#1}}
\def\nd*base{undefined}
\def\nd*opend[#1]#2{\nd*cmd\nd*clr\nd*openb\nd*guard{#1}#2}
\def\nd*close{\nd*cmd\nd*clr\nd*closeb}
\def\nd*guardd[#1]#2{\nd*guardb{#1}{#2}}

% Handling of optional arguments.

\def\nd*optarg#1#2#3{\ifx[#3\def\nd*c{#2#3}\else\def\nd*c{#2[#1]{#3}}\fi\nd*c}
\def\nd*optargg#1#2#3{\ifx[#3\def\nd*c{#1#3}\else\def\nd*c{#2{#3}}\fi\nd*c}

\def\nd*five#1{\nd*optargg{\nd*four{#1}}{\nd*two{#1}}}
\def\nd*four#1[#2]{\nd*optarg{0}{\nd*three{#1}[#2]}}
\def\nd*three#1[#2][#3]#4{\nd*optarg{}{#1[#2][#3]{#4}}}
\def\nd*two#1{\nd*three{#1}[\relax][]}

\def\nd*have{\nd*five{\nd*haved}}
\def\nd*hypo{\nd*five{\nd*hypod}}
\def\nd*open{\nd*optarg{}{\nd*opend}}
\def\nd*guard{\nd*optarg{1}{\nd*guardd}}

\def\nd*init{%
  \let\open\nd*open%
  \let\close\nd*close%
  \let\hypo\nd*hypo%
  \let\have\nd*have%
  \let\hypocont\nd*hypocont%
  \let\havecont\nd*havecont%
  \let\by\nd*by%
  \let\guard\nd*guard%
  \def\ii{\by{$\rightarrow$i}}%    %JD modification
  \def\ie{\by{$\rightarrow$e}}%    %JD modification
  \def\bii{\by{$\leftrightarrow$i}}%    %JD modification
  \def\bie{\by{$\leftrightarrow$e}}%    %JD modification
  \def\bi{\by{$\leftrightarrow$i}}%    %JD modification
  \def\be{\by{$\leftrightarrow$e}}%    %JD modification
  \def\Ai{\by{$\forall$i}}%
  \def\Ae{\by{$\forall$e}}%
  \def\Ei{\by{$\exists$i}}%
  \def\Ee{\by{$\exists$e}}%
  \def\ai{\by{$\wedge$i}}%
  \def\ae{\by{$\wedge$e}}%
  \def\aeone{\by{$\wedge$e$_1$}}%    %JD addition
  \def\aetwo{\by{$\wedge$e$_2$}}%    %JD addition
  \def\ai{\by{$\wedge$i}}%
  \def\aione{\by{$\wedge$i$_1$}}%    %JD addition for false proof questions
  \def\aitwo{\by{$\wedge$i$_2$}}%    %JD addition for false proof questions
  \def\ae{\by{$\wedge$e}}%
  \def\oi{\by{$\vee$i}}%
  \def\oione{\by{$\vee$i$_1$}}%    %JD addition
  \def\oitwo{\by{$\vee$i$_2$}}%    %JD addition
  \def\oe{\by{$\vee$e}}%
  \def\oeone{\by{$\vee$e$_1$}}%    %JD addition for false proof questions
  \def\oetwo{\by{$\vee$e$_2$}}%    %JD addition for false proof questions
  \def\ni{\by{$\neg$i}}%
  \def\ne{\by{$\neg$e}}%
  \def\be{\by{$\bot$e}}%
  \def\nne{\by{$\neg\neg$e}}%JD addition
  \def\nni{\by{$\neg\neg$i}}%JD addition
  \def\r{\by{R}}%
  \def\copy{\by{copy}}% %JD addition
  \def\pbc{\by{PBC}}%    %JD addition
  \def\mt{\by{MT}}%    %JD addition
  \def\lem{\by{LEM}}%    %JD addition
  \def\bn{\by{$\bot_\neg$}}%
  \def\tn{\by{$\top_\neg$}}%
  \def\premise{\by{premise}} % JD addition
  \def\assumption{\by{assumption}} % JD addition
  \def\propositional{\by{propositional}} % JD addition
  \def\Implied{\by{Implied}} % JD addition
  \def\Assignment{\by{Assignment}} % JD addition
  \def\IfStatement{\by{If-Statement}} % JD addition
  \def\IfStatementone{\by{If-Statement-1}} % JD addition
  \def\IfStatementtwo{\by{If-Statement-2}} % JD addition
  \def\PartialWhile{\by{Partial-While}} % JD addition
  \def\InvariantGuard{\by{Invariant Hyp. and Guard}} % JD addition
  \def\TotalWhile{\by{Total-While}} % JD addition
  \def\Ki{\by{$K$i}}%
  \def\Ke{\by{$K$e}}%
  \def\KT{\by{$KT$}}%
  \def\Kfour{\by{$K4$}}%
  \def\Kfive{\by{$K5$}}%
  \def\EKi{\by{$E$i}}%
  \def\EKe{\by{$E$e}}%
  \def\CKi{\by{$C$i}}%
  \def\CKe{\by{$C$e}}%
  \def\CK{\by{$CK$}}%
  \def\Cfour{\by{$C4$}}%
  \def\Cfive{\by{$C5$}}%
}

\newenvironment{nd}{\begingroup\nd*init\nd*beginc}{\nd*endc\endgroup}
\newenvironment{ndresume}{\begingroup\nd*init\nd*resumec}{\nd*endc\endgroup}

\catcode`\*=\nd*astcode

% End of file fitch.sty



\newcommand{\Z}{\mathbb{Z}}
\newcommand{\Q}{\mathbb{Q}}
\newcommand{\R}{\mathbb{R}}
\newcommand{\N}{\mathbb{N}}
\def\land{\wedge}
\def\lor{\vee}
\def\implies{\rightarrow}
\def\iff{\leftrightarrow}
\def\turn{\vdash}
\def\lrturn{\dashv\vdash}
\def\Cn{\text{Cn}}
\def\Th{\text{Th}}
\def\defeq{\stackrel{\rm def}{=}}

%%% The environment for providing answers to problems
\newenvironment{answer}%
{\par\noindent\textbf{Answer}\par\noindent}%
{}

\def\Sometime{\mathord{\mathsf{F}}}
\def\Forever{\mathord{\mathsf{G}}}
\def\Next{\mathord{\mathsf{O}}}
\def\NextX{\mathord{\mathsf{X}}}
\def\Until{\mathrel{\mathsf{U}}}
\def\Release{\mathrel{\mathsf{R}}}
\def\WeakUntil{\mathrel{\mathsf{W}}}
\def\Before{\mathrel{\mathsf{B}}}

\def\True{\mathord{\mathsf{true}}}

\def\All{\mathord{\mathsf{A}}}
\def\Exists{\mathord{\mathsf{E}}}
\def\Every{\mathord{\mathsf{E}}}

\def\True{\mathord{\mathtt{true}}}
\def\False{\mathord{\mathtt{false}}}

\def\If{\mathrel{\mathtt{if}}}
\def\Else{\mathrel{\mathtt{else}}}
\def\While{\mathrel{\mathtt{while}}}
\def\IfElse#1#2#3{\If #1 \ \{ #2 \} \Else \{ #3 \}}
\def\Whiledo#1#2{\While #1\ \{ #2 \}}
\def\Hcond#1{\llparenthesis #1 \rrparenthesis}
\def\Hoare#1#2#3{\Hcond{#1} \mathrel{#2} \Hcond{#3}}

\def\parmodels{\mathrel{\models_{\textup{par}}}}
\def\totmodels{\mathrel{\models_{\textup{tot}}}}
\def\parturn{\mathrel{\turn_{\textup{par}}}}
\def\totturn{\mathrel{\turn_{\textup{tot}}}}


\def\unityid{rsandil}

\begin{document}
\begin{center}
  {\LARGE CSC 503 Homework Assignment 7}\\[1pc]
  Out: September 28, 2015 \\
  Due: October 5, 2015 \\
  \unityid
\end{center}

\begin{enumerate}

\item \textbf{[30 points total]} Let $\phi_1$ and $\phi_2$ be the
  sentences
  \begin{eqnarray*}
    \phi_1 
    & =
    & \forall x \ \neg P(x,f(x)) \\
    \phi_2
    & = 
    & \forall x \forall y \forall z \ 
      P(x,y) \land P(y,z) \implies P(f(x),f(z))
  \end{eqnarray*}
  where $P$ is a predicate symbol of two arguments and $f$ is a
  function symbol of one argument.  Assume that $P$ and $f$ are the
  only nonlogical symbols in the language.

  \begin{itemize}

  \item[(a)] \textbf{[10 points]:} Give a formal interpretation $I$ of
    the language that makes $\phi_1$ true and $\phi_2$ false.
    
\begin{answer}
       Let the Interpretation $I$ from above such that p(a, b) means $ a \land b$.
       \begin{eqnarray*}
    \phi_1 
    & =
    & \forall x \ \neg P(x,f(x)) \\
    \end {eqnarray*}
       
     Let f(x) = $\neg x$
    
    \textbf{Check for $\phi_1$ :}
    
    $\forall x( x \land \neg x)$ is $False$ which means $\forall x P(x,f(x))$ is $False$. Thus $\forall x (\neg P(x,f(x)))$ is $True$
    
    Hence $\phi_1$ is \textbf{$True$}
    
   \textbf{Check for $\phi_2$ :}
    
   Left hand side: $ \forall x \forall y \forall z \ P(x,y) \land P(y,z) $ means $\forall x \forall y \forall z  (x \land y) \land (y \land z)$ 
    
    \textbf{If left hand side is True}, we have $(x \land y)$ as $True$ and $(y \land z)$ as $True$ which means x is $True$, y is $True$ and z is $True$
   
    
    \textbf{Right hand side}: P(f(x),f(z)) which means $f(x)\land f(z)$  
    
    Now since f(x) = $\neg x$  and f(z) = $\neg z$, $f(x)\land f(z)$ means $\neg x \land \neg z$.
    
    if left hand side is $True$, x and z is $True$ which means $\neg x$ and $\neg z$ both are $False$
    
    Thus $\neg x \land \neg z$ is $False$
    
    If left hand side is true, right hand side is being evaluated as false. Thus $\phi_2$ is \textbf{$False$}
          
\end{answer}
  \item[(b)] \textbf{[5 points]:} Briefly explain why $I$ makes
    $\phi_1$ true and makes $\phi_2$ false.
   \begin{answer} 
    p(x, y) means $x \land y$. Whenever y is a function of x taken in such a way that it is negation of x, p(x, y)
     will be False. Thus $\neg p(x, y)$ will be True making $\phi_1$ as $True$. In case of  $\phi_2$, left hand side if evaluates to True, we get x, y and z evaluated as $True$. if x and z are $True$, right hand side is evaluated to False, as shown above, Since $T \implies F$ is $F$. Thus $\phi_2$ evaluates to $False$.
      
\end{answer}
  \item[(c)] \textbf{[10 points]:} Give the formal definition of an
    interpretation $J$ that makes $\phi_1$ false and $\phi_2$ true.
    
    \begin{answer}
       Let the Interpretation $J$ be defined such that p(a, b) means  a \textless b.
       \begin{eqnarray*}
    \phi_1 
    & =
    & \forall x \ \neg P(x,f(x)) \\
    \end {eqnarray*}
    
    Let f(x) = x+1
    
   \textbf{Check for $\phi_1$ :}    
    
    $\forall x,(x  \textless  x+1$) is $True$. Thus $\forall x  P(x,f(x))$ is $True$
    
    
    Since $\forall x$ P(x,f(x)) is $True$, thus $\forall x,(\neg  P(x,f(x)))$ is $False $
    
    
    Hence $\phi_1$ is $False$
    
   \textbf{Check for $\phi_2$ :}
    
   \textbf{If Left hand side is $True$}: $ \forall x \forall y \forall z \ P(x,y) \land P(y,z)$ means $\forall x \forall y \forall z (x \textless y) \land (y \textless z)$ 
    
    which means  $x \textless  z $
    
   \textbf{ Right hand side}: P(f(x),f(z)) which means f(x)$\textless$ f(z)  
    
    Now if f(x) = x + 1  and f(z) = z + 1
    
    Thus f(x)$\textless$ f(z) means x + 1 $\textless $ z + 1. Thus  x $\textless $ z  which is $True$
    
    Thus if Left hand side is True, Right hand side also evaluates to $True$ . Since $T \implies T$ is $T$. Hence  $\phi_2$ is $True$.
    
    
       
\end{answer}

  \item[(d)] \textbf{[5 points]:} Briefly explain why $J$ makes
    $\phi_1$ false and makes $\phi_2$ true.
    \begin{answer}
     p(x, y) means x is less than y. Whenever y is a function of x taken in such a way that it is less than x, p(x, y)
     will be True. Thus $\neg p(x, y)$ will be False making $\phi_1$ $False$ for all x.
     
      In case of  $\phi_2$, if left hand side  evaluates to $True$, we get $x \textless z$. If  $x \textless z$, we get right hand side as True, as shown above as $T \implies T$ is $T$. Hence $\phi_2$ will evaluates to True.
    \end{answer}

  \end{itemize}
  
  

\item \textbf{[30 points total]} Apply the unification algorithm to
  each of the following sets.  For each set, at each step $i$, show
  (I) the disagreement of $S_i$, (II) the substitution $\sigma_i$ if
  there is one, or an explanation why there is no unifying
  substitution, (III) the result $S_{i+1}$ of applying $\sigma_i$ to
  $S_i$.  If the set unifies, show also (IV) the overall substitution
  $\sigma_0 \dots \sigma_k$ expressed as a single substitution, not as
  a composition.

For your reference, the algorithm to calculate the most general
unifier of a set $S$ of predicate expressions consists of the
following steps.
\begin{itemize}

\item Step 0:
  \begin{itemize}
  \item Set $S_0 = S$
  \item Set $\sigma_0 = \epsilon$
  \end{itemize}

\item Step $k+1$:
  \begin{itemize}
  \item If $|S_k| = 1$, return the product substitution $\sigma_0\cdots\sigma_k$
  \item If the disagreement set $D(S_k)$ contains both a variable
    $v$ and a term $t$ in which $v$ \emph{does not occur}, then
    \begin{itemize}
    \item Choose least such pair
    \item Set $\sigma_{k+1} = \{ t/v \}$
    \item Set $S_{k+1} = S_k \sigma_{k+1}$
    \item Proceed to step $k+2$
    \end{itemize}
  \item Otherwise, announce that $S$ has no unifier
  \end{itemize}
\end{itemize}

  In the following expressions, assume that $a,b,c$ are constant
  symbols, $f,g,h$ are function symbols, $P$ is a predicate symbol,
  and $u,v,w,x,y,z$ are variable symbols.

  \begin{enumerate}

  \item \textbf{[10 points]}
    $S = \{ P(b,y,f(y)), P(x,x,z) \}$
    
        \begin{answer}
		Initializing $\sigma_0$ to $\{\}$
		
		$S_0 = \{P(b, y, f(y)), P(x, x, z)\}$
		
		$D(S_0) = \{b, x\}$
		\bigskip
		
		$\sigma_1 = \{b/x\}$
		
		$S_1 = \{P(b, y, f(y)), P(b, b, z)\}$
		
		$D(S_1) = \{b, y\}$
		\bigskip
		
		$\sigma_2 = \{b/y\}$

		$S_2 = \{P(b, b, f(b)), P(b, b, z)\}$

		$D(S_2) = \{f(b), z\}$	
		
		$\sigma_3 = \{f(b)/z\}$
		
		$S_3 = \{P(b, b, f(b)), P(b, b, f(b))\}$
		
		$|S_3| = 1$	
	
		\bigskip
		
		$\sigma = \sigma_0 \cdot{} \sigma_1 \cdot{} \sigma_2 \cdot{} \sigma_3$
		
		$\sigma = \{\} \cdot{} \{b/x\}  \cdot{} \{b/y\} \cdot{} \{f(b)/z\}$
		
		$\sigma = \{b/x\}  \cdot{} \{b/y\} \cdot{} \{f(b)/z\}$
		
		$\sigma = \{b/x, b/y\} \cdot{} \{f(b)/z\}$
		
		$\sigma = \{b/x, b/y, f(b)/z\}$
		
		Unification is feasible for above $\sigma$.
    \end{answer}

  \item \textbf{[10 points]}
    $S = \{ P(x,x), P(y,g(h(y))) \}$
        \begin{answer}
		Initializing $\sigma_0$ to $\{\}$
		
		$S_0 = \{P(x, x), P(y, g(h(y)))\}$
		
		$D(S_0) = \{x, y\}$
		\bigskip
		
		$\sigma_1 = \{x/y\}$
		
		$S_1 = \{P(x, x), P(x, g(h(x)))\}$
		
		$D(S_1) = \{x, g(h(x))\}$
		
		Since g(h(x)) contains x,substitution is not possible.
		
		Unification is not feasible for above $\sigma$.
    \end{answer}
  \item \textbf{[10 points]}
    $S = \{ P(f(w,a),h(g(v),b)), P(f(w,w),h(x,y)), P(f(v,a),h(g(v),b)) \}$
      
      \begin{answer}
      
		Initializing $\sigma_0$ to $\{\}$
		
		$S_0 = \{P(f(w, a), h(g(v),b)), P(f(w, w), h(x, y)), P(f(v, a), h(g(v), b))\}$
		
		$D(S_0) = \{w, v\}$
		\bigskip
		
		$\sigma_1 = \{w/v\}$
		
		$S_1 = \{P(f(w, a), h(g(w),b)), P(f(w, w), h(x, y)), P(f(w, a), h(g(w), b))\}$
		
		$D(S_1) = \{a, w\}$
		\bigskip
		
		$\sigma_2 = \{a/w\}$

		$S_2 = \{P(f(a, a), h(g(a),b)), P(f(a, a), h(x, y)), P(f(a, a), h(g(a), b))\}$

		$D(S_2) = \{g(a), x\}$	
		
		$\sigma_3 = \{g(a)/x\}$
		
		$S_3 = \{P(f(a, a), h(g(a),b)), P(f(a, a), h(g(a), y)), P(f(a, a), h(g(a), b))\}$
		
		$D(S_3) = \{b, y\}$	
		
		$\sigma_4 = \{b/y\}$
		
		$S_4 = \{P(f(a, a), h(g(a),b)), P(f(a, a), h(g(a), b)), P(f(a, a), h(g(a), b))\}$
		
		$|S_4| = 1$	
	
		\bigskip
		
		$\sigma = \sigma_0 \cdot{} \sigma_1 \cdot{} \sigma_2 \cdot{} \sigma_3 \cdot{} \sigma_4$
		
		$\sigma = \{\} \cdot{} \{w/v\}   \cdot{} \{a/w\}  \cdot{} \{g(a)/x\} \cdot{} \{b/y\}$
		
		$\sigma = \{w/v\}   \cdot{} \{a/w\}  \cdot{} \{g(a)/x\} \cdot{} \{b/y\}$
		
		$\sigma = \{a/v, a/w\} \cdot{} \{g(a)/x\} \cdot{} \{b/y\}$
		
		$\sigma = \{a/v, a/w, g(a)/x\} \cdot{} \{b/y\}$
		
		$\sigma = \{a/v, a/w, g(a)/x, b/y\}$
		
		Unification is feasible for above $\sigma$.
    \end{answer}
  \end{enumerate}

\newpage

\item \textbf{[10 points]} Using the clauses
  \begin{center}
    \begin{tabular}{lll}
      Line & Clause & Justification \\ \hline
      1. & $\{ \neg P, Q \}$ & Given \\
      2. & $\{ \neg P, R \}$ & Given \\
      3. & $\{ \neg S, \neg T, U \}$ & Given \\
      4. & $\{ \neg U, \neg Q, V \}$ & Given \\
      5. & $\{ P \}$ & Given \\
      6. & $\{ S \}$ & Given \\
      7. & $\{ T \}$ & Given \\ \hline
    \end{tabular}
  \end{center}
  over the propositional symbols $P,Q,R,S,T,U,V$, give a resolution
  refutation proof of $V$ in either tree or linear form.
  \begin{answer}
  From logical point of view, we want to prove: $\{A1, A2, A3, A4, A5, A6, A7\} \turn V$ where
  
   A1: $\{ \neg P, Q \}$, A2: $\{ \neg P, R \}$, A3: $\{ \neg S, \neg T, U \}$, A4: $\{ \neg U, \neg Q, V \}$, A5: $\{ P \}$, A6: $\{ S \}$, A7: $\{ T \}$
   
  Thus By refutation, we need to check the consistency of:
  
  C = $\{A1, A2, A3, A4, A5, A6, A7\} \cup \{\neg V\}$
  
		\begin{enumerate}
		  \item $\{ \neg P, Q \}$
		  \item $\{ \neg P, R \}$
		  \item $\{ \neg S, \neg T, U \}$
		  \item $\{ \neg U, \neg Q, V \}$
		  \item $\{ P \}$
		  \item $\{ S \}$ 
		  \item $\{ T \}$ 
		  \item $\{\neg V\}$ 
		  \item $\{Q\}$  From Steps $1 + 5$
		  \item $\{\neg T, U\}$ From Steps $3 + 6$
		  \item $\{U\}$ From Steps $7 + 10$
		  \item $\{\neg Q, V\}$ From Steps $4 + 11$
		  \item $\{V\}$ From Steps $9 + 12$
		  \item $\square$ From Steps $8 + 13$
		  
		 \end{enumerate}
		 
		 Hence C is inconsistent by resolution/refutation so proof of V is established.
  
  \end{answer}
  

\item \textbf{[30 points]} Give a resolution refutation of the
  following set of clauses.  At each step, indicate the literals being
  resolved together and the substitutions being made.  You may find it
  helpful to standardize variables apart in each clause and at each
  step.
  \begin{center}
    \begin{tabular}{llll}
      Line & Clause & Justification & Substitution \\ \hline
      1. & $\{ P(a,u,f(h(u))), Q(u,a), R(h(b),b) \}$ & Given & \\
      2. & $\{ P(a,x,f(y)), P(a,z,f(h(b))), \neg R(y,z) \}$ & Given & \\
      3. & $\{ \neg P(a,w,f(h(b))), Q(x,a) \}$ & Given & \\
      4. & $\{ \neg R(h(b),w), Q(w,a) \}$ & Given & \\
      5. & $\{ \neg Q(v,a) \}$ & Given & \\ \hline
    \end{tabular}
  \end{center}
	\begin{answer}
		\textbf{Resolution} 
		
		\begin{enumerate}
		  \item $\{P(a, u, f(h(u))), Q(u, a), R(h(b), b)\}$
		  \item $\{P(a, x, f(y)), P(a, z, f(h(b))), \neg R(y, z)\}$
		  \item $\{\neg P(a, w, f(h(b))), Q(x, a)\}$
		  \item $\{\neg R(h(b), w), Q(w, a)\}$
		  \item $\{\neg Q(v, a)\}$
		  \item $\{\neg R(h(b), v)\}$ From Steps $4 + 5$. Substituting {v/w}
		  \item $\{\neg P(a, w, f(h(b)))\}$ From Steps $3 +  5$. Substituting {v/x}
		  \item $\{ P(a, v, f(h(v))), R(h(b), b)\}$ From Steps $1 + 5$. Substituting {v/u}
		  \item $\{ P(a, b, f(h(b)))\}$ From Steps $6 + 8$. Substituting {b/v}
		  \item $\square$ From $7 + 9$. Substituting $\{b/w\}$
		 \end{enumerate}
	\end{answer}
\end{enumerate}


\end{document}
