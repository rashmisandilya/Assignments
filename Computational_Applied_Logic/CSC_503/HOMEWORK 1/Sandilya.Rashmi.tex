\documentclass[12pt]{article}

% Package and macro definitions for CSC 503
% Originally prepared August 23, 2012 by Jon Doyle

%%% Page dimensions
\setlength{\oddsidemargin}{0in}
\setlength{\evensidemargin}{0in}
\setlength{\topmargin}{0in}
\setlength{\textheight}{9in}
\setlength{\textwidth}{6.5in}
\setlength{\headheight}{0in}
\setlength{\headsep}{0in}
\setlength{\footskip}{0.5in}

%%% Font and symbol definition packages
\usepackage{times} 
\usepackage{helvet} 
\usepackage{alltt}
\usepackage{amsfonts, amsmath, amsthm}
\usepackage{amssymb}
\usepackage{stmaryrd}

%%% TikZ diagramming package
\usepackage{tikz}
\usetikzlibrary{arrows,automata}

%%% The modified Sellinger fitch.sty file
\input{fitchhr.sty}

\newcommand{\Z}{\mathbb{Z}}
\newcommand{\Q}{\mathbb{Q}}
\newcommand{\R}{\mathbb{R}}
\newcommand{\N}{\mathbb{N}}
\def\land{\wedge}
\def\lor{\vee}
\def\implies{\rightarrow}
\def\iff{\leftrightarrow}
\def\turn{\vdash}
\def\lrturn{\dashv\vdash}
\def\Cn{\text{Cn}}
\def\Th{\text{Th}}
\def\defeq{\stackrel{\rm def}{=}}

%%% The environment for providing answers to problems
\newenvironment{answer}%
{\par\noindent\textbf{Answer}\par\noindent}%
{}

\def\Sometime{\mathord{\mathsf{F}}}
\def\Forever{\mathord{\mathsf{G}}}
\def\Next{\mathord{\mathsf{O}}}
\def\NextX{\mathord{\mathsf{X}}}
\def\Until{\mathrel{\mathsf{U}}}
\def\Release{\mathrel{\mathsf{R}}}
\def\WeakUntil{\mathrel{\mathsf{W}}}
\def\Before{\mathrel{\mathsf{B}}}

\def\True{\mathord{\mathsf{true}}}

\def\All{\mathord{\mathsf{A}}}
\def\Exists{\mathord{\mathsf{E}}}
\def\Every{\mathord{\mathsf{E}}}

\def\True{\mathord{\mathtt{true}}}
\def\False{\mathord{\mathtt{false}}}

\def\If{\mathrel{\mathtt{if}}}
\def\Else{\mathrel{\mathtt{else}}}
\def\While{\mathrel{\mathtt{while}}}
\def\IfElse#1#2#3{\If #1 \ \{ #2 \} \Else \{ #3 \}}
\def\Whiledo#1#2{\While #1\ \{ #2 \}}
\def\Hcond#1{\llparenthesis #1 \rrparenthesis}
\def\Hoare#1#2#3{\Hcond{#1} \mathrel{#2} \Hcond{#3}}

\def\parmodels{\mathrel{\models_{\textup{par}}}}
\def\totmodels{\mathrel{\models_{\textup{tot}}}}
\def\parturn{\mathrel{\turn_{\textup{par}}}}
\def\totturn{\mathrel{\turn_{\textup{tot}}}}




\begin{document}
Q1.

 a) Answer:

 p - Jack ran up the hill.
 
 q - Jill ran up the hill.
 
 $(p \land q)$
 \\*
 
 b) Answer:

 p - Real Madrid FC ran up the hill
 
  $p$
 \\*
 
 c) Answer:

 p - NC State has a red and white logo.
 
  $p$
  \\*


 d) Answer:

 P – Jack fell down, q – jack broke his crown.
 
 ($\neg p) \implies (\neg q)\ $
 \\*
 

 e) Answer:

 P – Jack fell down, q – jack broke his crown, r – Jill came tumbling after

 $ (p \land q)\land r$
 \\*


 
 Q2.
 
 a) $((\neg p)\lor q) \implies ((\neg(\neg q))\land(\neg r))$
 \\*
 
 b) $r \implies(((\neg q)\lor p)\implies (q \implies (\neg p)\lor r))$
 \\*
 
 
 Q3.
 
Let F be $((p \implies \neg q) \lor (p \land s) \implies r) \land \neg r$ and S(F) be the set of subformulas of F

Then

S(F) = {$((p \implies \neg q) \lor (p \land s) \implies r) \land \neg r$, $(p \implies \neg q) \lor (p \land s) \implies r$, $(p \implies \neg q) \lor (p \land s)$, $(p \implies \neg q)$, $(p \land s)$,$p$, $\neg q$, $s$,$\neg r$, $r$}
\\*
 

Q4.

The expression $p \lor q \land r$ is problematic because both $\land$ and $\lor$ have same connective precedence. This will result in conflict or disagreement if interpreted in different ways as follows:
\\*

First interpretation:
 $(p \lor q) \land r$
  
  	\begin{displaymath}
  		\begin{array}[t]{|c|c|c|c|c|} \hline
    		p & q & r & (p \lor q) & (p \lor q) \land r \\ \hline\hline
    		T & T & T & T & T \\ \hline
    		T & T & F & T & F \\ \hline
    		T & F & T & T & T \\ \hline
  	 		T & F & F & T & F \\ \hline
  	 		F & T & T & T & T \\ \hline
    		F & T & F & T & F \\ \hline
    		F & F & T & F & F \\ \hline
  	 		F & F & F & F & F \\ \hline
  		\end{array}
  	\end{displaymath}
  	
  	Second interpretation:
  	
  	$p \lor (q \land r)$
  	
  	\begin{displaymath}
  		\begin{array}[t]{|c|c|c|c|c|} \hline
    		p & q & r & (q \land r) & p \lor (q \land r) \\ \hline\hline
    		T & T & T & T & T \\ \hline
    		T & T & F & F & T \\ \hline
    		T & F & T & F & T \\ \hline
  	 		T & F & F & F & T \\ \hline
  	 		F & T & T & T & T \\ \hline
    		F & T & F & F & F \\ \hline
    		F & F & T & F & F \\ \hline
  	 		F & F & F & F & F \\ \hline
  		\end{array}
  	\end{displaymath}

Q5.

The complete truth table of
$((p \implies \neg q) \implies (q \implies p)$ can be demonstrated as:

\begin{center}
	\begin{tabular}{|c c c | c | c | c|}
	\hline
	$p$ & $q$ & $\neg q$ & $p \implies \neg q$ & $q \implies p$ & $(p \implies \neg q) \implies (q \implies p)$ \\ \hline
	T & T & F &F &T &T \\ \hline
	T & F & T &T &T &T \\ \hline
	F & T & F &F &F &F \\ \hline
	F & F & T &T &T &T \\ \hline
	
	\end{tabular}
	\end{center}
	
Q6.

The Truth table for the formula $(p \implies q) \lor (\neg q \land \neg r)$ can be demonstrated as :

\begin{center}
	\begin{tabular}{|c c c | c | c | c | c | c | }
	\hline
	$p$ & $q$ & $r$ & $p \implies q$ & $\neg q$ & $\neg r$ & $\neg q \land \neg r $ & $(p \implies q) \lor (\neg q \implies \neg r) $  \\ \hline
	T & T & T & T & F & F & F & T \\ \hline
	T & T & F & T & F & T & F & T \\ \hline
	T & F & T & F & T & F & F & \colorbox{red!50}{F} \\ \hline
	T & F & F & F & T & T & T & T \\ \hline
	F & T & T & T & F & F & F & T \\ \hline
	F & T & F & T & F & T & F & T \\ \hline
	F & F & T & T & T & F & F & T \\ \hline
	F & F & F & T & T & T & T & T \\ \hline
	\end{tabular}
	\end{center}

$(p \implies q) \lor (\neg q \land \neg r)$ , thus  is satisfiable as some interpretation (here all except one)  makes it true but is not valid since every interpretation does not make it true.
\\*


Q7.


The Truth table for $\neg p \implies (r \lor q), \neg q \land p \models p \implies q$ is:

\begin{center}
	\begin{tabular}{|c c c | c | c | c | c | c | c | }
	\hline
	$p$ & $q$ & $r$ & $\neg p$ & $\neg q $ & $r \lor q$ & $\neg p \implies(r \lor q)$ & $\neg q \land p $ & $p \implies q $   \\ \hline
    T & T & T & F & F & T & T & F & T\\ \hline
	T & T & F & F & F & T & T & F & T\\ \hline
	T & F & T & F & T & T & \colorbox{red!50}{T} & \colorbox{red!50}{T} & \colorbox{red!50}{F}\\ \hline
	T & F & F & F & T & F & \colorbox{red!50}{T} & \colorbox{red!50}{T} & \colorbox{red!50}{F}\\ \hline
	F & T & T & T & F & T & T & F & T\\ \hline
	F & T & F & T & F & T & T & F & T\\ \hline
	F & F & T & T & T & T & T & F & T\\ \hline
	F & F & F & T & T & F & F & F & T\\ \hline
	\end{tabular}
	\end{center}

The entailment claim is not true due to the valuation highlighted in the table since the truth values of the formulas to the left are T and truth values of the formula to the right are F.
\\*


Q8.

The Truth table for $\models (p \lor q) \land (\neg q \lor r) \implies (p \lor r)$ is:

\begin{center}
	\begin{tabular}{|c c c | c | c | c | c | c | c |}
	\hline
	$p$ & $q$ & $r$  & $\neg q$ &$ p \lor q$ & $\neg q \lor r $  & $(p \lor q) \land (\neg q \lor r)$ & $p \lor r$ & $(p \lor q) \land (\neg q \lor r) \implies (p \lor r)$\\ \hline
	
	T & T & T & F & T & T & T & T & T\\ \hline
	T & T & F & T & T & F & F & T & T\\ \hline
	T & F & T & F & T & T & T & T & T\\ \hline
	T & F & F & T & T & T & T & T & T\\ \hline
	F & T & T & F & T & T & T & T & T\\ \hline
	F & T & F & T & T & F & F & F & T\\ \hline
	F & F & T & F & F & T & F & T & T\\ \hline
	F & F & F & T & F & T & F & F & T\\ \hline

\end{tabular}
	\end{center}
From the truth table, it can be seen that the claim holds as it evaluates to T for every valuation.
\\*

\end{document}
